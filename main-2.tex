\documentclass{article}
\usepackage{circuitikz}

\title{ASSIGNMENT 9}
\author{Uttakarshika}
\date{December 16, 2020}

\begin{document}

\maketitle

\section{All logic gates shown in the figure have a propagation delay for 20ns. Let A=C=0 and B=1 until time t=0. At t=0, all inputs flip(i.e., A=C=1 and B=0) and remain in that state. for t greater than 0, output Z=1 for a duration of(in ns)-\\\ 
SOLUTION- \\\ Using the graphs and table provided, we come to the conclusion that-\\\\\\\
Z=1 FOR t=20ns-60ns\\\
BOOLEAN EXPRESSION  \overline{A\bar{B}}+A\bar{B}\bar{C}}

\begin{circuitikz}
\draw(0,-5) node[american not port,number inputs=1](P);
\draw(3,-3) node[american and port,number inputs=2](Q);
\draw(6,-4) node[american xor port,number inputs=2](R);
\draw(P.out) |- (Q.in 2);
\draw(Q.out) |- (R.in 1);

\node(x3) at (-1,-5){$B$}, \node(x2) at (1.3,-2.7){$A$}, \node(x1) at (4.3,-4.26){$C$},\node(x4) at (1,-5){$B'$}, \node(x5) at (3.3,-2.7){$X$},\node(x6) at (6.5,-4){$Z$};
\end{circuitikz}
\begin{center}
    \begin{tabular}{||c c c c c c c||}
    \hline
    time(t) & A & B & C & B' & X & Z\\[3ex]
    \hline\hline
    less than 0 & 0 & 1 & 0 & 0 & 0 & 0\\
    \hline
    0 & 1 & 0 & 1 & 0 & 0 & 0\\
    \hline
    20 & 1 & 0 & 1 & 1 & 0 & 1\\
    \hline
    40 & 1 & 0 & 1 & 1 & 1 & 1\\
    \hline
    60 & 1 & 0 & 1 & 1 & 1 & 0\\
    \hline
    80 & 1 & 0 & 1 & 1 & 1 & 0\\
    \hline
    \end{tabular}
    
    
\end{center}

\\{KEY FOR GRAPHS-\\ALONG X-AXIS- 1 UNIT=20ns\\\ALONG Y-AXIS- 1 UNIT=1}

\begin{tikzpicture}[domain=0:4]
\draw[very thin,color=gray] (-0.1,-1.1) grid (3.9,3.9);

\draw[->] (-0.2,0) -- (4.2,0) node[right] {$TIME(t)$};
\draw[->] (0,-1.2) -- (0,4.2) node[above] {$A$};

\draw[color=red]  (0,1)--(4,1)      node[right] {$y=1$};

\end{tikzpicture}


\begin{tikzpicture}[domain=0:4]
\draw[very thin,color=gray] (-0.1,-1.1) grid (3.9,3.9);

\draw[->] (-0.2,0) -- (4.2,0) node[right] {$TIME(t)$};
\draw[->] (0,-1.2) -- (0,4.2) node[above] {$B$};

\draw[color=green]  (0,0)--(4,0) node[right];

\end{tikzpicture}

\begin{tikzpicture}[domain=0:4]
\draw[very thin,color=gray] (-0.1,-1.1) grid (3.9,3.9);

\draw[->] (-0.2,0) -- (4.2,0) node[right] {$TIME(t)$};
\draw[->] (0,-1.2) -- (0,4.2) node[above] {$C$};

\draw[color=green]  (0,1)--(4,1)       node[right] ;

\end{tikzpicture}



\begin{tikzpicture}[domain=0:4]
\draw[very thin,color=gray] (-0.1,-1.1) grid (3.9,3.9);

\draw[->] (-0.2,0) -- (4.2,0) node[right] {$TIME(t)$};
\draw[->] (0,-1.2) -- (0,4.2) node[above] {$B'$};

\draw[color=red]  (0,0)--(1,0)--(1,1)--(4,1)     node[right] ;

\end{tikzpicture}

\begin{tikzpicture}[domain=0:4]
\draw[very thin,color=gray] (-0.1,-1.1) grid (3.9,3.9);

\draw[->] (-0.2,0) -- (4.2,0) node[right] {$TIME(t)$};
\draw[->] (0,-1.2) -- (0,4.2) node[above] {$X$};

\draw[color=red]  (0,0)--(2,0)--(2,1)--(4,1)     node[right] ;

\end{tikzpicture}


\begin{tikzpicture}[domain=0:4]
\draw[very thin,color=gray] (-0.1,-1.1) grid (3.9,3.9);

\draw[->] (-0.2,0) -- (4.2,0) node[right] {$TIME(t)$};
\draw[->] (0,-1.2) -- (0,4.2) node[above] {$Z$};

\draw[color=red]  (0,0)--(1,0)--(1,1)--(3,1)--(3,0)--(4,0)     node[right] ;

\end{tikzpicture}




\end{document}
